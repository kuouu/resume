\newcommand{\resumeProjEntry}[3]{
    \vspace{5pt}\item
      \begin{tabular*}{0.97\textwidth}{l@{\extracolsep{\fill}}r}
        \textbf{#1} #2 & \small #3\\
      \end{tabular*}\vspace{-5pt}
}

% \resumeProjEntry
% {Real-time Messaging App}
% {[Express.js, Socket.io]|[\href{https://fse-chat.glitch.me}{website}]}
% {July 2023 - July 2023}
% {
%     \resumeItemListStart
%         \resumeItem
%         {Developed a chatting app using object-oriented programming in JavaScript, emphasizing encapsulation and inheritance}
%         \resumeItem
%         {Applied the Block Element Modifier (BEM) methodology in frontend development to ensure a modular CSS approach}
%         \resumeItem
%         {Incorporated Prisma (ORM) with the SQLite database, ensuring type-safe DB interactions and streamlined data management}
%     \resumeItemListEnd
% }

\resumeProjEntry
{Shopping Mall Navigator (Meichu Hackathon)}
{[Chatbot, React.js, FastAPI]|[\href{https://github.com/OscarShiang/meichu-stack}{code}, \href{https://line-tv.kuouu.tw}{website}]}
{Oct 2021 - Oct 2021}
{
    \resumeItemListStart
        \resumeItem
        {Built a customer-centric system aimed at recommending stores of interest, elevating the overall shopping experience}
        \resumeItem
        {Integrated LINE BEACON technology to detect customer entry, initiating an immediate in-store guide via a chatbot}
        \resumeItem
        {Designed a graphical dashboard interface for store owners, facilitating chatbot design and marketing data monitoring}
    \resumeItemListEnd
}

% \resumeProjEntry
% {Smart Home App}
% {[React.js, Flask, Web Notification API]|[\href{https://github.com/kuouu/smart-room-frontend}{code}]}
% {Sep 2020 - Jun 2021}
% {
%     \resumeItemListStart
%         \resumeItem
%         {Implemented push notifications via service workers, ensuring real-time app updates and enhancing user experience}
%         \resumeItem
%         {Developed a web dashboard for efficient smart home appliance scheduling, resulting in a 20\% reduction in energy consumption}
%     \resumeItemListEnd
% }

% \resumeProjEntry
% {Fight COVID-19 - Tower Defense Game}
% {[Java, Unified Modeling Language (UML)]|[\href{https://github.com/kuouu/fight-covid19}{code}]}
% {Feb 2020 - Jun 2020}
% {
%     \resumeItemListStart
%         \resumeItem
%         {Developed a tower defense game to raise awareness about the importance of taking preventive measures against COVID-19}
%         \resumeItem
%         {Utilized Java and object-oriented programming (OOP) principles to create a robust and scalable game architecture}
%         \resumeItem
%         {Utilized UML to map out the game's core functionalities, ensuring a clear understanding of the system's flow and interactions}
%     \resumeItemListEnd
% }

% \resumeProjEntry
% {Smart Audio Mixer}
% {[C++, JUCE]|[\href{https://github.com/kuouu/smart-audio-mixer}{code}]}
% {Sep 2021 - Jan 2022}
% {
%     \resumeItemListStart
%         \resumeItem
%         {Led a three-member team to build a modern audio editor with third-party plugin support}
%         \resumeItem
%         {Reconstructed the user interface of the application to improve usability and match different operating systems' needs}
%     \resumeItemListEnd
% }

\resumeProjEntry
{Data Station (Tainan Innovation Datathon)}
{[Flask, PostgreSQL, React.js, Apache ECharts]}
{Sep 2022 - Sep 2022}
{
    \resumeItemListStart
        \resumeItem
        {Analyzed and integrated data from three distinct enterprise systems, including the smart meters, accommodation sales, and power backup capacity to empower companies to achieve sustainable operations and ESG responsibility}
        \resumeItem
        {Developed a backend service using Flask and PostgreSQL to reduce reserve capacity demand by 40\% through demand response and enable accommodation providers to offer corresponding discounts to consumers}
        \resumeItem
        {Created a visual data analytics platform with Apache ECharts, earning future sustainability award among 20 teams}
    \resumeItemListEnd
}

\resumeProjEntry
{Ninjutsu Mudra Recognition}
{[PyTorch, OpenCV, PyQt]|[\href{https://docs.google.com/presentation/d/1uDKyUByoG-cLUBiKg4SqwVUj7In7w9YE/preview}{slides}]}
{Sep 2021 - Jan 2022}
{
    \resumeItemListStart
        \resumeItem
        {Led a team of 10 to develop the Ninjutsu Mudra Recognition Project with VGG-16, achieving an 85\% accuracy}
        \resumeItem
        {Enhanced efficiency with EfficientNet to enable real-time recognition capabilities, and integrate the model into desktop application using OpenCV and PyQt}
    \resumeItemListEnd
}
